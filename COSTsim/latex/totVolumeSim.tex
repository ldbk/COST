\HeaderA{totVolumeSim}{Estimation of total volume of discards or/and landings (weight, number or number-at-length) of simulated data sets}{totVolumeSim}
\keyword{methods}{totVolumeSim}
\begin{Description}\relax
This function is the equivalent to \code{totVolume} for 'simDataCons' class objects. It estimates total volume of discards or/and 
landings (weight, number or number-at-length) based on various raising methods for simulated data sets.
\end{Description}
\begin{Usage}
\begin{verbatim}
totVolume(dbeOutputSim,simObject,...)
\end{verbatim}
\end{Usage}
\begin{Arguments}
\begin{ldescription}
\item[\code{dbeOutputSim}] A \emph{dbeOutputSim} object. All necessary information for calculation process are taken in the first slots (species, catch category,...). See \emph{dbeObject} method for object initialization.
\item[\code{simObject}] A \emph{simDataCons} object matching 'dbeOutputSim' specifications.
\item[\code{...}] Further arguments such as:
\item[type] Specification of the raising method : \code{"trip"} (default value) for raising by trip, \code{"fo"} for raising by fishing operations, 
\code{"fd"} for raising by fishing days,\code{"landings"} for ratio-to-total landings raising method, and \code{"time"} for ratio-to-fishing duration 
raising method.
\item[val] Estimated parameter. To be chosen between \code{"weight"} (default value), \code{"number"} and \code{"nAtLength"}.
\item[sampPar] logical specifying if given species is considered to be automatically sampled during the sampling process (default value is \code{TRUE}).
\item[landSpp] character vector describing the species considered in the 'volume of landings' variable if chosen raising method is ratio-to-landings (see 'clObject' description).  

\end{ldescription}
\end{Arguments}
\begin{Value}
An updated object of class dbeOutputSim.
\end{Value}
\begin{Author}\relax
Dorleta Garcia <dgarcia@azti.es>
\end{Author}
\begin{References}\relax
Vigneau, J. (2006)          
\emph{Raising procedures for discards : Sampling theory (Toward agreed methodologies for calculating precision in the discard programmes)}. Working document in support of PGCCDBS (Rostock, 2006).
\end{References}
\begin{SeeAlso}\relax
\code{\LinkA{totVolume}{totVolume}}
\end{SeeAlso}

