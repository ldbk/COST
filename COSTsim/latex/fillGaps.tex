\HeaderA{fillGaps}{Method for completing gaps in simulated data sets}{fillGaps}
\keyword{methods}{fillGaps}
\begin{Description}\relax
This method completes dbeSimObj class objects with zeros in non sampled age and length dimensions 
in order to have objects of the same dimension that can be compared
\end{Description}
\begin{Usage}
\begin{verbatim}
fillGaps(dbeSimObj, ageMin, ageMax, lenMin, lenMax)
\end{verbatim}
\end{Usage}
\begin{Arguments}
\begin{ldescription}
\item[\code{dbeSimObj}] A dbeOutputSim object
\item[\code{ageMin}] Numeric indicating the minimum age sampled
\item[\code{ageMax}] Numeric indicating the maximum age sampled
\item[\code{lenMin}] Numeric indicating the minimum length sampled
\item[\code{lenMax}] Numeric indicating the maximum length sampled
\end{ldescription}
\end{Arguments}
\begin{Value}
It returns the same object but with the age and length dimensions completed with zeros
\end{Value}
\begin{Author}\relax
Dorleta Garcia \email{dgarcia@azti.es}
\end{Author}

