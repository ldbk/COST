\HeaderA{simDataVal}{Class "simDataVal"}{simDataVal}
\keyword{classes}{simDataVal}
\begin{Description}\relax
The simDataVal-class is equivalent in structure to simData but stores validated forms of the simulated data sets
\end{Description}
\begin{Section}{Objects from the Class}
The creator function \code{simDataVal} can be called to create objects from this class.
\end{Section}
\begin{Section}{Slots}
\Tabular{lrl}{
\bold{slot} & \bold{class} & \bold{description} \\
\bold{\code{desc}} & \code{character} & object description \\
\bold{\code{species}} & \code{character} & species description. Recall of SL\$spp and SL\$sex \\
\bold{\code{samples}} & \code{list} & each element in the list is a costDataVal object representing a simulated data set \\
\bold{\code{initial.fit}} & \code{list} &  \\
\bold{\code{setup.args}} & \code{list} & set up parameters \\
\bold{\code{burnin}} & \code{numeric} &  \\
\bold{\code{nmcmc}} & \code{numeric} &  \\
\bold{\code{l.int}} & \code{numeric} &  \\
\bold{\code{Int}} & \code{list} &  \\
\bold{\code{Slp}} & \code{list} &  \\
\bold{\code{landings}} & \code{numeric} &  \\
\bold{\code{nHaul}} & \code{integer} &  \\
\bold{\code{nseas}} & \code{numeric} &  \\
}
\end{Section}
\begin{Author}\relax
Dorleta Garcia \email{dgarcia@azti.es}
\end{Author}
\begin{Examples}
\begin{ExampleCode}
showclass("simDataVal")
\end{ExampleCode}
\end{Examples}

